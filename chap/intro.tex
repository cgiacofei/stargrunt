\chapter{Introduction}
\begin{quotation}
\begin{itshape}
Sergeant Thrasher is moving cautiously through the brush with his leading squad; Private Funk is on point (for the third day running, something to do with what happened to the Officers' latrines...), behind him are Thrasher, four more riflemen and the squad machinegunner. Suddenly a little amber light flashes in Funk's HUD visor, centred on a clump of rocks and bushes to his right; at the same time he hears the chatter of an automatic weapon, and dives flat. There are yells and curses over the squad comm, and a single short scream. Thrasher's voice overrides the babble on the comm: ``Squad, target! Designate green - range two hundred - free fire!''

Funk peers up over a nice large rock he has found, and his HUD gives him a green flasher where the Sergeant is designating the target point. He hears Anderson let rip with the SAW from somewhere behind him, along with sporadic shots from the rest of the squad; realising that no one (especially Thrasher) can see him, he slides down behind the rock again - no point in letting the Euries know where he is by shooting at them....

The enemy fire seems to have stopped. Someone is yelling over the comm: ``Shit, Sarge - Harry's copped one, we need a C-Vac for Chrissake......'' 
\end{itshape}
\end{quotation}

Grunts, Squaddies, Pongoes, Footsloggers, call 'em what you will - the infantry have been there as long as armies have been around. Sure, they've got lots of shiny new toys, missiles that're smarter than they are and all the rest - but they're still the ones that get left up to their knees in mud, getting shelled and sniped at, holding the line like no tank or gunship can.

These rules are for infantry actions in a science-fiction environment; whether this is set twenty years from now or two thousand is up to you - as far as your little metal soldiers are concerned they'll still get the rough end of it..... 

What style of game is STARGRUNT II? Well, for a start it is not meant to be ``HOW WAR WILL BE FOUGHT IN THE FAR FUTURE''. I don't KNOW how war will be fought in the far future (If you DO know how it will actually happen, could you let me know next week's lottery numbers while you're about it...?) but I could hazard a guess that a lot of it will involve a few automated drones shooting at each other over a virtually empty battlefield - not very inspiring as a miniatures game!

No, what we are doing here is creating an environment for SF mini atures games that has a similar style to those portrayed in Combat SF novels, films and TV series, most of these have their roots firmly in present-day military concepts, tactics and organisation - which is why the Colonial Marines in the Aliens movies look (and act) just like present-day US Marines, and the ``boot camp'' sequences in Heinlein's classic Starship Troopers would be familiar to any recruit from WW1 onwards.

I guess this is because these are concepts that we all understand and are reasonably familiar with, so they can be taken as read and not get in the way of the action and storytelling - we can all relate to the situations and things ``feel'' right, so the whole thing becomes believable and accepted.

Consider that although technology and tactics may change, basic human nature doesn't. Look at the average soldier of today, compared to the average soldier of, say, Napoleonic times. Sure, today's grunt is better trained, better fed, and MUCH better equipped - but he is still just a man with all the same hopes, emotions and vulnerabilities - and he still regards Officers with distrust and Sergeants with a complex mixture of fear, loathing and grudging respect. We can safely assume that our SF troops will behave very much like infantry have done throughout history, which gives us a starting point we can identify with.

So, what we have tried to do with STARGRUNT Il is to produce a system for simulating SF actions where the ordinary soldiers are not too unlike those tramping across the battlefields of yesterday or today - they may carry a Gauss rifle rather than an M16 or a Brown Bess, but they are still the Poor Bloody Infantry and still think the same way. Their supporting tanks may hover or float on grav fields, but they are still tanks, artillery and air support still fulfil similar battlefield roles to their twentieth century counterparts.

Realistic? Probably not. Believable? Maybe. Fun? We sincerely hope so!

So, what are you waiting for? Lets get out there and kill something. I love the smell of Plasma in the morning...... 

\section{DESIGNER'S NOTES}
Some of you may be thinking ``this is STARGRUNT II - so what was STARGRUNTI?'' Well, the original STARGRUNT was a small-press rules booklet we first published six years ago, as an attempt to do an SF combat system that actually made the troops react like ``real'' soldiers rather than little tin clairvoyant superheroes. SG has sold a good few hundred copies in its life and acquired a keen following of players, but due to its format has always been of fairly limited availability. In SGII we have taken the opportunity to completely revise the game and bring the mechanisms up to date, while keeping (we hope) the essential feel that made SG so popular.

We have tried to produce a game that encourages the players to THINK TACTICALLY. The rules on Confidence, Motivation, Suppression etc. are designed in such a way that a simple frontal assault (you know, the ``line 'em all up at the baseline and advance across the table'' approach so common (sadly) with many wargamers) will in all likelihood NOT work - at least, not unless you have MASSIVE force superiority in which case you haven't thought out the scenario properly!

In SGII, you can't just rely on your firepower and some lucky die rolling to win the game for you - you actually have to work for it.

As with the original, SGII is a GENERIC rules set - it is designed to be tailored to whatever forces, figures and background you wish to use. We have provided our own ``official'' (in the loosest sense of the word) background in a separate section, so that those of you who wish to use it may do so without it intruding too much on the main generic rules.

If you wish to use your own background or one lifted from a film, book or TV series, then you will need to adapt some parts of the rules to fit the particular hardware and style of action from your chosen source. ``Realism'', in terms of Science Fiction games, means being as faithful as possible to your source material, whatever that may be.

Players of our preceding rules set, DIRTSIDE II (1/300 SF armour rules) will notice immediately that we have retained many of the basic mechanisms and principles from these rules in STARGRUNT II, this is partly to give a common factor to the two games which will enable players to transfer easily from one to the other, and partly because the principles worked very well in DSII - as the old saying goes, ``If it ain't broke, don't fix it.....''

Read the rules through, then use them as you wish - you've paid out the money, and it's now your game as much as ours!


